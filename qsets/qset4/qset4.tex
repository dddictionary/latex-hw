\documentclass[letterpaper,11pt]{article}
\usepackage{natbib}
\bibliographystyle{unsrtnat}
\usepackage{tabularx} % extra features for tabular environment
\usepackage{amsmath}
\usepackage{amssymb}
\usepackage{amsfonts}
\usepackage{mathtools}
\usepackage{graphicx} % takes care of graphic including machinery
\usepackage[margin=1in,letterpaper]{geometry} % decreases margins
%\usepackage{cite} % takes care of citations
\usepackage[final]{hyperref} % adds hyper links inside the generated pdf file
\graphicspath{ {./images/} }

\DeclarePairedDelimiter\floor{\lfloor}{\rfloor}  % improve math presentation
\hypersetup{
	colorlinks=true,       % false: boxed links; true: colored links
	linkcolor=blue,        % color of internal links
	citecolor=blue,        % color of links to bibliography
	filecolor=magenta,     % color of file links
	urlcolor=blue         
}
%+++++++++++++++++++++++++++++++++++++++

\begin{document}

\title{Discrete Math Question Set 4}
\author{Abrar Habib}
\date{November 28, 2022}
\maketitle

\begin{enumerate}
    \item [1.] Let P(n) be the statement that a postage of n cents can be formed using just 3-cent stamps and 5-cent
    stamps. The parts of this exercise outline the alternative form of mathematical induction proof that P(n) is
    true for n $\geq$ 8.
    \begin{enumerate}
        \item [a.] Show that the statements P(8), P(9) and P(10) are true, completing the basis step of the proof.
        \item [] I can make 8 by adding 3+5. I can make 9 by adding 3 + 3 + 3. I can make 10 by addomg 5+5. Therefore, P(8), P(9), and P(10) are true.
        \item [b.] What is the inductive hypothesis of the proof?
        \item [] The inductive hypothesis of the proof is that P(n) is true for 8 $\leq$ n $\leq$ k, where k $\geq$ 10.
        \item [c.] What do you need to prove in the inductive step?
        \item [] We need to prove that the inductive case P(k+1) is true.
        \item [d.] Complete the inductive step for k $\geq$ 10.
        \item [] If k $\geq$ 10, then k+1 = (k-2) + 3 $\therefore $ k-2 = 8. By induction hypothesis, we have proven P(k-2) to be true. This represents the fact that a package of k-2 cents can be paid by using a 3-cent and 5-cent page. If we add one more 3-cent stamp, we can pay a package of k+1 cents. $\therefore$ P(k+1) is true.
        \item [e.] Explain why these steps show that this statement is true whenever n $\geq$ 8.
        \item [] By principle of strong induction, the statement is true for all n $\geq$ 8. Since we have proven P(8), P(9), and P(10), any n-cent postage $\geq$ 8 can be attained by adding a certain number of 3-cent packages.
    \end{enumerate}
    \item [2.] Find the unique solution for each of the following recurrence relations.
    \begin{enumerate}
        \item [a.] $a_{n+1}-1.5a_n=0, n \geq 0$
        \item [] $a_n=1.5a_{n-1} = 1.5^2a_{n-2}$ There is a pattern here such that $a_n = 1.5^n a_0$
        \item [b.] $4a_n - 5a_{n-1} = 0, n \geq 1$
        \item [] $4a_n = 5a_{n-1}$
        \item [] $a_n = \frac{5}{4}a_{n-1} = \frac{5}{4}^2a_{n-2}$ There is another pattern here such that $a_n = \frac{5}{4}^na_0$
    \end{enumerate}
    \newpage
    \item [3.] Solve the following recurrence relations. (In case of complex roots, only general solution is enough.)
    \begin{enumerate}
        \item [1.] $2a_{n+2} - 11a_{n+1} +5a_n = 0, n \geq 0, a_0 = 2, a_1 = -8$
        \item [] Characteristics equation: $2r^2+11r+5 = 0$
        \item [] $2x^2-11x+5 = 0 \implies (x-5)(2x-1) = 0 \implies x = \frac{1}{2}$ and $x={5}$
        \item [] General solution: $a_n = ax_1^n + bx_2^n$
        \item [] We are given $a_0 = 2 = a(\frac{1}{2})^0 + b(5)^0$
        \item [] $2 = a+b \implies a=2-b$
        \item [] We are given $a_1 = -8 = a(\frac{1}{2})^1 + b(5)^1$
        \item [] $-8 = \frac{1}{2}a + 5b \implies -8 = \frac{1}{2} (2-b) + 5b \implies -8 = 1-\frac{b}{2}+5b$
        \item [] $-16 = 2-b+10b \implies -18 = 9b \implies b = -2$
        \item [] $2=a+b \implies 2=a-2 \implies a=4$
        \item [] Therefore the general equation for this recurrence relation is $a_n = 4(\frac{1}{2})^n-2(5)^n$
        \item[2.] $a_{n+2} + a_n = 0, n \geq 0, a_0 = 0, a_1 = 3$
        \item[] Characteristics equation = $r^2+1 = 0$
        \item[] Characteristics root = $r_1 = i, r_2=-i$
        \item[] General solution: $a_n = ax_1^n + bx_2^n$
        \item[] $a_0 = a(i)^0+b(-i)^0$W With the given $a_0 = 0 = a+b \implies a = -b$
        \item[] $a_1 = 3 = a(i)^1 + b(-i)^1 \implies 3 = ai -bi \implies 3 = -bi-bi \implies b = -\frac{3}{2i}$
        \item[] Therefore, $a = \frac{3}{2i}$
        \item[] General equation for this recurrence relation is $a_n = \frac{3}{2i}(i)^n - \frac{3}{2i}(-i)^n$     
    \end{enumerate} 
    \item[4.] Using mathematical induction, prove that if A1, A2, · · · , An are subsets of a universal set U, then
    $$\overline{\bigcup_{k=1}^{n}A_k} = \bigcap_{k=1}^{n}\overline{A_k}$$
    CAN YOU GO OVER THIS IN CLASS PLEASE? I AM HAVING TROUBLE WITH IT!!!
    \newpage
    \item[5.] How many different strings can be made from the letters in ORONO using some or all of the letters?
    \item[] If we use 1 letter each we only get 1 string: ORN of which can be arranged 3 ways.
    \item[] If we use 2 letters, we can get: OO,OR,ON,RN. We can arrange those (1+2+2+2 = 7 different ways.)
    \item[] If we use 3 letters, we can get: OOO,OOR,OON,ORN. We can arrange those (1+3+3+6 = 13 different ways.)
    \item[] If we use 4 letters, we can get: OOOO,OOOR,OORN. We can arrange those (4+4+4 choose 2 = 20)
    \item[] IF we use 5 letters, we can get: OOORN. We can arrange this 5 choose 3 ways which is 20. If we add all these combinations together, we get 3+7+13+20+20=63 as our final answer.
    \item[6.]  What is the coefficient of $x^{101}y^{99}$ in the expansion of $(2x-3y)^{200}?$
    \item[] $(a+b)^n = \sum_{k=0}^{n} {n \choose k}a^kb^{n-k} a=2x, b=-3y, n=200, k=101$. With this we get that the coefficient will be: $${{200} \choose {101}}$$ The entire term of the expansion will be $${{200} \choose {101}} 2x^{101}-3y^{99}$$
    \item[7.] How many arrangements of the letters in MISSISSIPPI have no consecutive S's?
    \item[] $$\frac{{8 \choose 4}7!}{4!2!} = 7350$$
    \item[] This is because we have 7 letters total with 8 letters remaining. We want to fill in those 8 gaps with 4 S's, which can be done in 8C4 ways. The rest of the letters can be arranged in $\frac{7!}{4!2!}$ ways.
    \item[8.]
    \begin{enumerate}
        \item [a.] Fifteen points, no three of which are collinear, are given on a plane. How many lines do they determine?
        \item [] A line can be made with onyl two points. So we only need to find how many ways two pair two points from the total of 15. This gives us $15 \choose 2$ possibilites.
        \item [b.] Twenty-five points, no four of which are coplanar, are given in space. How many triangles do they determine?
        How many planes? How many tetrahedra (pyramidlike solids with four triangular faces)?
        \item [] They make $25 \choose 3$ triangles since it only takes 3 points to make a triangle. 
        \item [] They also make $25 \choose 3$ planes becuase you also only need 3 points for a plane.
        \item [] They make $25 \choose 4$ tetrahedra because we only need 4 points to make one. 
    \end{enumerate}  
    \newpage
    \item [9.] Sixteen people are to be seated at two circular tables, one of which seats 10 while the other seats six. How
    many different seating arrangements are possible?
    \item[] There are $16 \choose 10$ ways to choose how many of those people will sit at the first table. 
    \item[] Then we need to find how many ways to sit those 10 people in the first table, this is given by $9!$. Lastly, the number of ways to sit people in the last table is given by $5!$. In total we have $${16 \choose 10} 9! 6!$$
    \item[10.] Count the solutions to inequality $x_1+x_2+x_3 \leq 11$, knowing $x_1, x_2, x_3 \geq 0$are non-negative integers. 
    \item[]  This is a counting multisets equation (check chapter 10.9.1). $${n+m-1} \choose {m-1}$$ 
    We have $n = 11, m = 4$ so our answer is $${{11+4-1} \choose {4-1}} = 364$$
\end{enumerate}


\end{document}
