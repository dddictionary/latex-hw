%++++++++++++++++++++++++++++++++++++++++
% Don't modify this section unless you know what you're doing!
\documentclass[letterpaper,11pt]{article}
\usepackage{natbib}
\bibliographystyle{unsrtnat}
\usepackage{tabularx} % extra features for tabular environment
\usepackage{amsmath}
\usepackage{amsfonts}
\usepackage{mathtools}
\usepackage{graphicx} % takes care of graphic including machinery
\usepackage[margin=1in,letterpaper]{geometry} % decreases margins
%\usepackage{cite} % takes care of citations
\usepackage[final]{hyperref} % adds hyper links inside the generated pdf file
\DeclarePairedDelimiter\floor{\lfloor}{\rfloor}  % improve math presentation
\hypersetup{
	colorlinks=true,       % false: boxed links; true: colored links
	linkcolor=blue,        % color of internal links
	citecolor=blue,        % color of links to bibliography
	filecolor=magenta,     % color of file links
	urlcolor=blue         
}
%+++++++++++++++++++++++++++++++++++++++
\begin{document}

\title{Discrete Math Question Set 2}
\author{Abrar Habib}
\date{October 11, 2022}
\maketitle


\begin{enumerate}
    \item[1.] Determine whether each of these functions is a bijection from $\mathbb{R}$  to $\mathbb{R}$.
    \begin{enumerate}
      \item[a)] $f(x) = 2x + 1$
      \item[] To prove this function is bijective, we need to prove it is injective and surjective.
      
      A function is injective if f(a) = c and f(b) = c and a = b. Therefore, we get
      
      $c=2a+1$ \& $c = 2b +1$
      \begin{equation}
        \begin{aligned}
            & 2a+1 = 2b+1 &\\
            & 2a = 2b &\\
            & a = b
        \end{aligned}
      \end{equation} 
      Since $f(a) = f(b)$, and we proved that a = b, this function is injective. Now we must prove this function is surjunctive.
      \begin{equation}
        \begin{aligned}
            & f(x) = y &\\
            & 2x+1 = y &\\
            & x = \frac{y-1}{2} &\\
            & y = 2x+1 &\\
            & y = 2(\frac{y-1}{2}) + 1 &\\
            & y = y
        \end{aligned}
      \end{equation} 
      Because $f(x) = y$, this function is surjective. From our previous comclusions, it is also injective, and therefore also Bijective.

      \item[b)] $f(x) = x^2+1$ 
      \item[] Use same logic as before.
      \begin{equation}
        \begin{aligned}
            & a^2+1 = b^2+1 &\\
            & a^2 = b^2 &\\
            & \pm a = \pm b
        \end{aligned}
      \end{equation} 
      Therefore, this function is injective.

      For this function to be surjective, we need to prove f(x) = y.

      \begin{equation}
        \begin{aligned}
            f(x) &= y &\\
            x^2+1 &= y &\\
            x^2 &= y-2 &\\
            x &= \sqrt{y-2} &\\
            f(x) &= f(\sqrt{y-2}) &\\
            f(\sqrt{y-2}) &= {\sqrt{y-2}}^2 +1 &\\
            f(\sqrt{y-2}) &= y-2 + 1 &\\
            f(\sqrt{y-2}) &= y+1 &\\
        \end{aligned}
      \end{equation} 
      Since $f(x)\neq y$, this function is not surjective and therefore not bijective.
      % \newpage
      \item[c)] $f(x) = \frac{x^2+1}{x^2+2}$
      \item[] Use the same logic for proving injective-ness.
      \begin{equation}
        \begin{aligned}
          & \frac{a^2+1}{a^2+2} = c &\\
          & \frac{b^2+1}{b^2+2} = c &\\
          & \frac{a^2+1}{a^2+2} = \frac{b^2+1}{b^2+2} &\\
          & (a^2+1)(b^2+2) = (a^2+2)(b^2+1)  &\\
          & a^2b^2+2a^2+b^2+2 = a^2b^2+a^2+2b^2+2 &\\
          & a^2 = b^2 &\\
          & \pm a = \pm b
        \end{aligned}
      \end{equation} 
      This function is therefore not injective. For example. if we make $x = \pm 1$, we get the same output of $\frac{1}{2}$.
      This already makes this function not bijective.
    \end{enumerate}
    % \newpage
    \item [2.] Let $f : \mathbb{R} \rightarrow \mathbb{R}$ be defined by $f(x) = \lfloor x \rfloor $, the greatest integer in $x$. Find $f^-1(B)$ for the following subset B of $\mathbb{R}$.
    $B = [0,2)$
    \begin{equation}
      f^{-1}([0,2))  = \{ x \in \mathbb{R}| \floor{x} = 0 \} \cup \{ x x \in \mathbb{R} | \floor{x} = 1 \}
    \end{equation} 
    The floor function does not have an inverse. The floor function is also defined to take in only integers. I.E. $f(\mathbb{R}) = \mathbb{Z}$. Therefore, the 
    only values which I can find the inverse of are 0 and 1 (2 is not included in the set B). The "inverse" of this function is any x whose floor is either 0 or 1 and therefore is what makes answer a disjoint set. No floor(x) is equal to 2 different numbers. 
    \newpage
    \item[3.] Does the formula $f(x) = \frac{1}{x^2-2}$ define a function $f : \mathbb{R} \to \mathbb{R}$? What about $f : \mathbb{Z} \to \mathbb{R}?$
    \begin{enumerate}
      \item [a)] $f : \mathbb{R} \to \mathbb{R}$
      The function will be undefined if the denominator is equal to 0.
      \begin{equation}
        \begin{aligned}
          &x^2-2 = 0 &\\
          &x^2 = 2 &\\
          & x = \pm \sqrt{2}   
        \end{aligned}
      \end{equation}
      At $\pm \sqrt{2}$, the function is not well defined as there is a hole there.

      \item [b)] $f : \mathbb{R} \to \mathbb{Z}$
      Since the only number that makes the function undefined, $\pm \sqrt{2}$, is a real number, it is not part of the domain, which is $\mathbb{Z}$, and therefore makes the function well defined.
    \end{enumerate}
    \item[4.] For each of the following functions $f : \mathbb{Z} \to \mathbb{Z}$, determine whether the function is one-to-one and whether it is onto. If the function is not onto, determine the range of $f(\mathbb{Z})$.
    \begin{enumerate}
      \item [a)] $f(x) = -x+5$
      To prove if this function is one to one, we can check to see if two different inputs a and b give us the same output.
      \begin{equation}
        \begin{aligned}
          -a+5 &= c &\\
          -b+5 &= c &\\
          -a+5 &= -b+5 &\\
          -a& = -b &\\
          a& = b
        \end{aligned}
      \end{equation}
      Since we determined that a is equal to b, this function is one-to-one.

      To determine if a function is onto, we need to find its inverse (if it has one) and plug it back to the equation. It will be onto if $f(x) = y$.
      
      Take any $y \in \mathbb{Z}$. Also $x \in \mathbb{Z}$.

      \begin{equation}
        \begin{aligned}
          f(x) &= y &\\
          -x+5 &= y &\\
          (x &=5-y) \in \mathbb{Z}&\\
          f(5-y) &= -(5-y)+5 &\\
          f(5-y) &= -5+y+5 &\\
          f(5-y) &= y
        \end{aligned}
      \end{equation}
      This function is onto as well and its domain is $\mathbb{R}$. We made $f(x) = y$ with $y \in \mathbb{R}$ and proved the first equality. 
      \item [b)] $f(x) = x^2$ Once again, we can use the same principles from before.
      \begin{equation}
        \begin{aligned}
          a^2 &= c &\\
          b^2 &= c &\\
          a^2 &= b^2 &\\
          \pm a &= \pm b
        \end{aligned}
      \end{equation}
      This function is not one-to-one because of the plus minus. We can also logically see that if $x = \pm 1$ both output 1.
      
      This fucntion is also not onto because it does not map $f : \mathbb{R} \to \mathbb{R}$. There is no integer such that $x^2 = -1$.
      The domain therefore for this function is $\mathbb{Z^+}$ because the input domain was $\mathbb{Z}$ and we determined that the function has no input (within the domain) whose output is negative.
    
    \end{enumerate}
  \end{enumerate}

\end{document}
